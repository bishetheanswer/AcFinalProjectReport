\section{Internal organization/structure (datapath)}
The internal structure is the one pictured in Figure \ref*{cortexA72conf}. And the details of the blocks are described in section 2.1.1 of source \cite{cortexA72manual}.
\subsection*{Fetch, decode and dispatch units}
\subsubsection*{Instruction Fetch}
Up to 3 instructions are fetched from the L1 instruction memory. The cache has 48KB of size in a 3-way associativity configuration with 64-bits-long cache lines; and optionally, parity bits for error-checking in the data and tag. 
In order to facilitate address prediction and translation, the architecture provides this features: 
\begin{itemize}
	\item {For fast virtual address translation, a Translation Lookaside Buffer (TLB) is with support for pages of sizes 4KB, 64KB, and  1MB.} \label{tlb}
	\item {2-level dynamic Branch Target Buffer (BTB). A static predictor is also provided.}
\end{itemize}


\subsubsection*{Instruction decode}
Types of instructions able to be decoded by the CPU (including)\cite{cortexA72manual}:

\begin{itemize}
	\item {A32: 32-bit wide, aligned on 4-byte boundaries. Both A-profile and R-profile are supported (application and real-time applications respectively) It is the most supported ISA. 
		} \cite{a32instr}
	\item {T32: mixed 32 and 16-bit length instruction set, aiming at lost memory footprint and cost. 
		It's the only ISA supported by type M instructions (specific to microcontrollers)\cite{t32instr}} 
	\item {A64. Instructions semantics similar to A32 and T32, with 31 64-bit general purpose registers; independent from Program Counter and Stack Pointer, and hard-coded zero register.} \cite{a64instr}
	
	This unit performs register renaming to facilitate out-of-order execution; thus removing both WAW and WAR hazards.
\end{itemize}


\subsubsection*{Instruction dispatch}
The dispatch unit controls when the instructions that are decoded are ready to be issued, and afterwards, when the the results are retired. \cite{cortexA72manual}

\subsection*{Functional units for execution and memory access}
This set of functional units is composed by the units for operating with integers, floating point values, and memory accesses. \cite{cortexA72manual}

\subsubsection*{Integer execute}
\begin{itemize}
	\item 2 ALU pipelines.
	\item Integer multiply-accumulate and ALU pipelines.
	\item Iterative integer dividing unit.
	\item Unit for resolution and prediction of branches.
	\item Result forwarding routes.
\end{itemize}

\subsubsection*{Load/Store unit}
This unit is responsible of accessing L1 memory cache and service memory coherency requests with the second cache level (L2).

L1 is a 32KB 2-way associative cache with 64-bits-long lines, and can optionally implement Error Correction Code (ECC)\footnote{The difference between using parity bits in memories and using ECC memory is that, while parity bits helps \textbf{detecting} errors; ECC memory can both \textbf{detect} and \textbf{repair} those errors} for every 32 bits. Apart from the aforementioned TLB (see \nameref{tlb} ), this unit also uses automatic pre-fetching targeting the L1 data cache and L2. \cite{cortexA72manual}

\subsubsection*{Level-2 of cache}
Services the both the data and instruction L1 in case of cache miss, and keeps coherence. For that, it keeps a duplicate of the tag fields of each L1 cache. 
Its size is either 12KB, 1MB, 2MB, or 4MB; with a 16-way associativity with optional ECC data per 64 bits; and a 1024-entry TLB \textbf{for each processor}.
This cache, as explained in the section 7.2 of \cite{cortexA72manual} has a \textit{dirty} bit per line of cache (That is, per 64 bits).

\subsubsection*{Advanced SIMD and floating-point}
It gives support to ARMv8 advanced SIMD (Simple Input, multiple data) and floating-point operations. Optionally, it is up to the implementation to include an optional cryptography engine, albeit with contractual rights. \cite{cortexA72manual}

\subsubsection*{GIC CPU interface}
This unit delivers interrupts to the processor. Examples of interruptions: device signals (device ready, read/write completed, error; battery and power), clock IRQ (for scheduling), exceptions. \cite{cortexA72manual}

\subsection*{Interrupts and debug}


\subsubsection*{Generic Timer}
Provides with the ability to trigger interrupts and trigger events. \cite{cortexA72manual}

\subsubsection*{Debug and trace}

\begin{itemize}
	\item {ARMv8 debug architecture with slave architecture with debug registers. It provides the programmer with registers for discerning the interrupts \textbf{source}, \textbf{behavior} to apply, and \textbf{routing} to other processors.}
	\item {Performance Monitor Unit: creates statistic data from \textbf{runtime} status of the CPU and the memory}
	\item {Embedded Trace Macrocells: performance of real-time instruction flow tracing.}
	
	\cite{cortexA72manual}
\end{itemize}
\section{Cache memory}
\begin{tabular}{|c|c|c|c|}
	\hline 
	Cache level & Sizes available & Associativity & Cache line length \\ 
	\hline 
	Level 1 - Data & 32KB \textbf{per core} & 2-way set assoc. & 64 bits \\ 
	\hline 
	Level 1 - Instruction & 48KBs & 3-way set assoc. &  64 bits \\ 
	\hline 
	Level 2 & 12KB, 1MB, 2MB, or 4MB & 16-way associativity & 64 bits \\ 
	\hline	
\end{tabular} 


Writes to main memory use as policy \textbf{ Write-back-write-allocate}. This means that updating MM only takes place when cache is \textit{dirty} and prompt to be discarded, and not on MM directly; and upon a miss in cache, the block is brought to the upper levels and allocated there for it to be read or written.
\cite{cortexA72manual} 6.4.1
\section{I/O support}